%%%%%%%%%%%%%%%%%%%%%%%%%%%%%%%%%%%%%%%%%%%%%%%%%%%%%%%%%%%%%%%%%%%%%%%%%%%%%%%
%%% This tex file defines the shapes necessary for inclusion as tex macros. %%%
%%%%%%%%%%%%%%%%%%%%%%%%%%%%%%%%%%%%%%%%%%%%%%%%%%%%%%%%%%%%%%%%%%%%%%%%%%%%%%%

% Requires PGF >= v 1.18.

% This tex file defines the shapes necessary for inclusion as tex macros.
% Simply copy and paste this code in to your preamble and call the commands
% wherever you wish to display a glycan symbol.
%
% Alternatively, download this .tex file somewhere, and use %%%%%%%%%%%%%%%%%%%%%%%%%%%%%%%%%%%%%%%%%%%%%%%%%%%%%%%%%%%%%%%%%%%%%%%%%%%%%%%
%%% This tex file defines the shapes necessary for inclusion as tex macros. %%%
%%%%%%%%%%%%%%%%%%%%%%%%%%%%%%%%%%%%%%%%%%%%%%%%%%%%%%%%%%%%%%%%%%%%%%%%%%%%%%%

% Requires PGF >= v 1.18.

% This tex file defines the shapes necessary for inclusion as tex macros.
% Simply copy and paste this code in to your preamble and call the commands
% wherever you wish to display a glycan symbol.
%
% Alternatively, download this .tex file somewhere, and use %%%%%%%%%%%%%%%%%%%%%%%%%%%%%%%%%%%%%%%%%%%%%%%%%%%%%%%%%%%%%%%%%%%%%%%%%%%%%%%
%%% This tex file defines the shapes necessary for inclusion as tex macros. %%%
%%%%%%%%%%%%%%%%%%%%%%%%%%%%%%%%%%%%%%%%%%%%%%%%%%%%%%%%%%%%%%%%%%%%%%%%%%%%%%%

% Requires PGF >= v 1.18.

% This tex file defines the shapes necessary for inclusion as tex macros.
% Simply copy and paste this code in to your preamble and call the commands
% wherever you wish to display a glycan symbol.
%
% Alternatively, download this .tex file somewhere, and use %%%%%%%%%%%%%%%%%%%%%%%%%%%%%%%%%%%%%%%%%%%%%%%%%%%%%%%%%%%%%%%%%%%%%%%%%%%%%%%
%%% This tex file defines the shapes necessary for inclusion as tex macros. %%%
%%%%%%%%%%%%%%%%%%%%%%%%%%%%%%%%%%%%%%%%%%%%%%%%%%%%%%%%%%%%%%%%%%%%%%%%%%%%%%%

% Requires PGF >= v 1.18.

% This tex file defines the shapes necessary for inclusion as tex macros.
% Simply copy and paste this code in to your preamble and call the commands
% wherever you wish to display a glycan symbol.
%
% Alternatively, download this .tex file somewhere, and use \input{/path/to/snfg.tex}
% You'll need to load the Tikz package in your preamble yourself if you do this.

% Author: Joe R. J. Healey
% Version: 1.0
% Email: J.R.J.Healey@warwick.ac.uk

% Changelog:
%  2018-05-16
%    > Initial creation and definition of some initial macros.
%     - Colours (RGB matches to NCBI documentation)
%     - Hexoses and N-acetyl hexoses

% To Do:
%  2018-05-16
%    > Replace hard coded values with variables and defaults.
%    > If the above is done, define a single macro for each glycan class (e.g. pentose)
%      and then make sub-macros of it for each individual sugar (e.g. \Ara{#1,#2,#3}

%%%%%%%%%%%%%%%%%%%%%%%%%%%%%%%%%%%%%%%%%%%%%%%%%%%%%%%%%%%%%%%%%%%%%%%%%%%%%%%%

\usepackage{tikz} 
\usetikzlibrary{angles,
                quotes,
                calc,
                positioning,
                shapes,
                arrows}


% Symbol Nomenclature for Glycans

% Colors (capitalised first letter to distinguish from inbuilt "green" etc.
\definecolor{Blue}{RGB}{0, 134, 248}
\definecolor{Green}{RGB}{0, 166, 91}
\definecolor{Yellow}{RGB}{255, 213, 65}
\definecolor{Orange}{RGB}{255, 118, 50}
\definecolor{Pink}{RGB}{255, 136, 192}
\definecolor{Purple}{RGB}{158, 58, 247}
\definecolor{LightBlue}{RGB}{129, 205, 230}
\definecolor{DarkBrown}{RGB}{148, 98, 41}
\definecolor{Red}{RGB}{255, 0, 24}

% Hexoses  (To do: replace some of the hard-coded values with vars and defaults.)
\newcommand{\Hexose}{\raisebox{0.5pt}{\tikz{\node[draw,line width=0.3mm, scale=1.1, circle, fill=white](){};}}}
\newcommand{\Glc}{\raisebox{0.5pt}{\tikz{\node[draw,line width=0.3mm, scale=1.1, circle, fill=Blue](){};}}}
\newcommand{\Man}{\raisebox{0.5pt}{\tikz{\node[draw,line width=0.3mm, scale=1.1, circle, fill=Green](){};}}}
\newcommand{\Gal}{\raisebox{0.5pt}{\tikz{\node[draw,line width=0.3mm, scale=1.1, circle, fill=Yellow](){};}}}
\newcommand{\Gul}{\raisebox{0.5pt}{\tikz{\node[draw,line width=0.3mm, scale=1.1, circle, fill=Orange](){};}}}
\newcommand{\Alt}{\raisebox{0.5pt}{\tikz{\node[draw,line width=0.3mm, scale=1.1, circle, fill=Pink](){};}}}
\newcommand{\All}{\raisebox{0.5pt}{\tikz{\node[draw,line width=0.3mm, scale=1.1, circle, fill=Purple](){};}}}
\newcommand{\Tal}{\raisebox{0.5pt}{\tikz{\node[draw,line width=0.3mm, scale=1.1, circle, fill=LightBlue](){};}}}
\newcommand{\Ido}{\raisebox{0.5pt}{\tikz{\node[draw,line width=0.3mm, scale=1.1, circle, fill=DarkBrown](){};}}}
%return

% N-Acetyl Hexoses
\newcommand{\HexNAc}{\raisebox{0.5pt}{\tikz{\node[draw,line width=0.3mm, scale=1.1, regular polygon, regular polygon sides=4, fill=white](){};}}}
\newcommand{\GlcNAc}{\raisebox{0.5pt}{\tikz{\node[draw,line width=0.3mm, scale=1.1, regular polygon, regular polygon sides=4, fill=Blue](){};}}}
\newcommand{\ManNAc}{\raisebox{0.5pt}{\tikz{\node[draw,line width=0.3mm, scale=1.1, regular polygon, regular polygon sides=4, fill=Green](){};}}}
\newcommand{\GalNAc}{\raisebox{0.5pt}{\tikz{\node[draw,line width=0.3mm, scale=1.1, regular polygon, regular polygon sides=4, fill=Yellow](){};}}}
\newcommand{\GulNAc}{\raisebox{0.5pt}{\tikz{\node[draw,line width=0.3mm, scale=1.1, regular polygon, regular polygon sides=4, fill=Orange](){};}}}
\newcommand{\AltNAc}{\raisebox{0.5pt}{\tikz{\node[draw,line width=0.3mm, scale=1.1, regular polygon, regular polygon sides=4, fill=Pink](){};}}}
\newcommand{\AllNAc}{\raisebox{0.5pt}{\tikz{\node[draw,line width=0.3mm, scale=1.1, regular polygon, regular polygon sides=4, fill=Purple](){};}}}
\newcommand{\TalNAc}{\raisebox{0.5pt}{\tikz{\node[draw,line width=0.3mm, scale=1.1, regular polygon, regular polygon sides=4, fill=LightBlue](){};}}}
\newcommand{\IdoNAc}{\raisebox{0.5pt}{\tikz{\node[draw,line width=0.3mm, scale=1.1, regular polygon, regular polygon sides=4, fill=DarkBrown](){};}}}
%return

% Hexosamine
% Half filled shapes are complicated, this will appear in a later update.

% Heuronate
% Half filled shapes are complicated, this will appear in a later update.

% Deoxyhexose (Note, numbers are not allowed in TeX macro command names, thus 6dTal becomes dTal)
\newcommand{\Deoxyhexose}{\raisebox{0.5pt}{\tikz{\node[draw, line width=0.3mm, scale=1.1, regular polygon, regular polygon sides=3,fill=white](){};}}}
\newcommand{\Qui}{\raisebox{0.5pt}{\tikz{\node[draw, line width=0.3mm, scale=1.1, regular polygon, regular polygon sides=3,fill=Blue](){};}}}
\newcommand{\Rha}{\raisebox{0.5pt}{\tikz{\node[draw, line width=0.3mm, scale=1.1, regular polygon, regular polygon sides=3,fill=Green](){};}}}
\newcommand{\dGul}{\raisebox{0.5pt}{\tikz{\node[draw, line width=0.3mm, scale=1.1, regular polygon, regular polygon sides=3,fill=Orange](){};}}}
\newcommand{\dAlt}{\raisebox{0.5pt}{\tikz{\node[draw, line width=0.3mm, scale=1.1, regular polygon, regular polygon sides=3,fill=Pink](){};}}}
\newcommand{\dTal}{\raisebox{0.5pt}{\tikz{\node[draw, line width=0.3mm, scale=1.1, regular polygon, regular polygon sides=3,fill=LightBlue](){};}}}
\newcommand{\Fuc}{\raisebox{0.5pt}{\tikz{\node[draw, line width=0.3mm, scale=1.1, regular polygon, regular polygon sides=3,fill=Red](){};}}}
%return

% DeoxyhexNAc

% Di-deoxyhexose

% Pentose
\newcommand{\Pentose}{\raisebox{0pt}{\tikz{\node[draw, line width=0.3mm, scale=0.9, star, star points=5, star point ratio=2, fill=white](){};}}}
\newcommand{\Ara}{\raisebox{0pt}{\tikz{\node[draw, line width=0.3mm, scale=0.9, star, star points=5, star point ratio=2, fill=Green](){};}}}
\newcommand{\Lyx}{\raisebox{0pt}{\tikz{\node[draw, line width=0.3mm, scale=0.9, star, star points=5, star point ratio=2, fill=Yellow](){};}}}
\newcommand{\Xyl}{\raisebox{0pt}{\tikz{\node[draw, line width=0.3mm, scale=0.9, star, star points=5, star point ratio=2, fill=Orange](){};}}}
\newcommand{\Rib}{\raisebox{0pt}{\tikz{\node[draw, line width=0.3mm, scale=0.9, star, star points=5, star point ratio=2, fill=Pink](){};}}}
%return

% Deoxynonulosonate
\newcommand{\Deoxynonulosonate}{\raisebox{0pt}{\tikz{\node[draw, line width=0.3mm, scale=1.1, diamond, fill=white](){};}}}
\newcommand{\Kdn}{\raisebox{0pt}{\tikz{\node[draw, line width=0.3mm, scale=1.1, diamond, fill=Green](){};}}}
\newcommand{\NeuAc}{\raisebox{0pt}{\tikz{\node[draw, line width=0.3mm, scale=1.1, diamond, fill=Purple](){};}}}
\newcommand{\NeuGc}{\raisebox{0pt}{\tikz{\node[draw, line width=0.3mm, scale=1.1, diamond, fill=LightBlue](){};}}}
\newcommand{\Neu}{\raisebox{0pt}{\tikz{\node[draw, line width=0.3mm, scale=1.1, diamond, fill=DarkBrown](){};}}}
\newcommand{\Sia}{\raisebox{0pt}{\tikz{\node[draw, line width=0.3mm, scale=1.1, diamond, fill=Red](){};}}}
%return

% Di-Deoxynonulosonate

% Unknown

% Assigned
\newcommand{\Assigned}{\raisebox{0.5pt}{\tikz{\node[draw, line width=0.3mm, scale=1.1, regular polygon, regular polygon sides=5,fill=white](){};}}}
\newcommand{\Api}{\raisebox{0.5pt}{\tikz{\node[draw, line width=0.3mm, scale=1.1, regular polygon, regular polygon sides=5,fill=Blue](){};}}}
\newcommand{\Fru}{\raisebox{0.5pt}{\tikz{\node[draw, line width=0.3mm, scale=1.1, regular polygon, regular polygon sides=5,fill=Green](){};}}}
\newcommand{\Tag}{\raisebox{0.5pt}{\tikz{\node[draw, line width=0.3mm, scale=1.1, regular polygon, regular polygon sides=5,fill=Yellow](){};}}}
\newcommand{\Sor}{\raisebox{0.5pt}{\tikz{\node[draw, line width=0.3mm, scale=1.1, regular polygon, regular polygon sides=5,fill=Orange](){};}}}
\newcommand{\Psic}{\raisebox{0.5pt}{\tikz{\node[draw, line width=0.3mm, scale=1.1, regular polygon, regular polygon sides=5,fill=Pink](){};}}}
%return


% You'll need to load the Tikz package in your preamble yourself if you do this.

% Author: Joe R. J. Healey
% Version: 1.0
% Email: J.R.J.Healey@warwick.ac.uk

% Changelog:
%  2018-05-16
%    > Initial creation and definition of some initial macros.
%     - Colours (RGB matches to NCBI documentation)
%     - Hexoses and N-acetyl hexoses

% To Do:
%  2018-05-16
%    > Replace hard coded values with variables and defaults.
%    > If the above is done, define a single macro for each glycan class (e.g. pentose)
%      and then make sub-macros of it for each individual sugar (e.g. \Ara{#1,#2,#3}

%%%%%%%%%%%%%%%%%%%%%%%%%%%%%%%%%%%%%%%%%%%%%%%%%%%%%%%%%%%%%%%%%%%%%%%%%%%%%%%%

\usepackage{tikz} 
\usetikzlibrary{angles,
                quotes,
                calc,
                positioning,
                shapes,
                arrows}


% Symbol Nomenclature for Glycans

% Colors (capitalised first letter to distinguish from inbuilt "green" etc.
\definecolor{Blue}{RGB}{0, 134, 248}
\definecolor{Green}{RGB}{0, 166, 91}
\definecolor{Yellow}{RGB}{255, 213, 65}
\definecolor{Orange}{RGB}{255, 118, 50}
\definecolor{Pink}{RGB}{255, 136, 192}
\definecolor{Purple}{RGB}{158, 58, 247}
\definecolor{LightBlue}{RGB}{129, 205, 230}
\definecolor{DarkBrown}{RGB}{148, 98, 41}
\definecolor{Red}{RGB}{255, 0, 24}

% Hexoses  (To do: replace some of the hard-coded values with vars and defaults.)
\newcommand{\Hexose}{\raisebox{0.5pt}{\tikz{\node[draw,line width=0.3mm, scale=1.1, circle, fill=white](){};}}}
\newcommand{\Glc}{\raisebox{0.5pt}{\tikz{\node[draw,line width=0.3mm, scale=1.1, circle, fill=Blue](){};}}}
\newcommand{\Man}{\raisebox{0.5pt}{\tikz{\node[draw,line width=0.3mm, scale=1.1, circle, fill=Green](){};}}}
\newcommand{\Gal}{\raisebox{0.5pt}{\tikz{\node[draw,line width=0.3mm, scale=1.1, circle, fill=Yellow](){};}}}
\newcommand{\Gul}{\raisebox{0.5pt}{\tikz{\node[draw,line width=0.3mm, scale=1.1, circle, fill=Orange](){};}}}
\newcommand{\Alt}{\raisebox{0.5pt}{\tikz{\node[draw,line width=0.3mm, scale=1.1, circle, fill=Pink](){};}}}
\newcommand{\All}{\raisebox{0.5pt}{\tikz{\node[draw,line width=0.3mm, scale=1.1, circle, fill=Purple](){};}}}
\newcommand{\Tal}{\raisebox{0.5pt}{\tikz{\node[draw,line width=0.3mm, scale=1.1, circle, fill=LightBlue](){};}}}
\newcommand{\Ido}{\raisebox{0.5pt}{\tikz{\node[draw,line width=0.3mm, scale=1.1, circle, fill=DarkBrown](){};}}}
%return

% N-Acetyl Hexoses
\newcommand{\HexNAc}{\raisebox{0.5pt}{\tikz{\node[draw,line width=0.3mm, scale=1.1, regular polygon, regular polygon sides=4, fill=white](){};}}}
\newcommand{\GlcNAc}{\raisebox{0.5pt}{\tikz{\node[draw,line width=0.3mm, scale=1.1, regular polygon, regular polygon sides=4, fill=Blue](){};}}}
\newcommand{\ManNAc}{\raisebox{0.5pt}{\tikz{\node[draw,line width=0.3mm, scale=1.1, regular polygon, regular polygon sides=4, fill=Green](){};}}}
\newcommand{\GalNAc}{\raisebox{0.5pt}{\tikz{\node[draw,line width=0.3mm, scale=1.1, regular polygon, regular polygon sides=4, fill=Yellow](){};}}}
\newcommand{\GulNAc}{\raisebox{0.5pt}{\tikz{\node[draw,line width=0.3mm, scale=1.1, regular polygon, regular polygon sides=4, fill=Orange](){};}}}
\newcommand{\AltNAc}{\raisebox{0.5pt}{\tikz{\node[draw,line width=0.3mm, scale=1.1, regular polygon, regular polygon sides=4, fill=Pink](){};}}}
\newcommand{\AllNAc}{\raisebox{0.5pt}{\tikz{\node[draw,line width=0.3mm, scale=1.1, regular polygon, regular polygon sides=4, fill=Purple](){};}}}
\newcommand{\TalNAc}{\raisebox{0.5pt}{\tikz{\node[draw,line width=0.3mm, scale=1.1, regular polygon, regular polygon sides=4, fill=LightBlue](){};}}}
\newcommand{\IdoNAc}{\raisebox{0.5pt}{\tikz{\node[draw,line width=0.3mm, scale=1.1, regular polygon, regular polygon sides=4, fill=DarkBrown](){};}}}
%return

% Hexosamine
% Half filled shapes are complicated, this will appear in a later update.

% Heuronate
% Half filled shapes are complicated, this will appear in a later update.

% Deoxyhexose (Note, numbers are not allowed in TeX macro command names, thus 6dTal becomes dTal)
\newcommand{\Deoxyhexose}{\raisebox{0.5pt}{\tikz{\node[draw, line width=0.3mm, scale=1.1, regular polygon, regular polygon sides=3,fill=white](){};}}}
\newcommand{\Qui}{\raisebox{0.5pt}{\tikz{\node[draw, line width=0.3mm, scale=1.1, regular polygon, regular polygon sides=3,fill=Blue](){};}}}
\newcommand{\Rha}{\raisebox{0.5pt}{\tikz{\node[draw, line width=0.3mm, scale=1.1, regular polygon, regular polygon sides=3,fill=Green](){};}}}
\newcommand{\dGul}{\raisebox{0.5pt}{\tikz{\node[draw, line width=0.3mm, scale=1.1, regular polygon, regular polygon sides=3,fill=Orange](){};}}}
\newcommand{\dAlt}{\raisebox{0.5pt}{\tikz{\node[draw, line width=0.3mm, scale=1.1, regular polygon, regular polygon sides=3,fill=Pink](){};}}}
\newcommand{\dTal}{\raisebox{0.5pt}{\tikz{\node[draw, line width=0.3mm, scale=1.1, regular polygon, regular polygon sides=3,fill=LightBlue](){};}}}
\newcommand{\Fuc}{\raisebox{0.5pt}{\tikz{\node[draw, line width=0.3mm, scale=1.1, regular polygon, regular polygon sides=3,fill=Red](){};}}}
%return

% DeoxyhexNAc

% Di-deoxyhexose

% Pentose
\newcommand{\Pentose}{\raisebox{0pt}{\tikz{\node[draw, line width=0.3mm, scale=0.9, star, star points=5, star point ratio=2, fill=white](){};}}}
\newcommand{\Ara}{\raisebox{0pt}{\tikz{\node[draw, line width=0.3mm, scale=0.9, star, star points=5, star point ratio=2, fill=Green](){};}}}
\newcommand{\Lyx}{\raisebox{0pt}{\tikz{\node[draw, line width=0.3mm, scale=0.9, star, star points=5, star point ratio=2, fill=Yellow](){};}}}
\newcommand{\Xyl}{\raisebox{0pt}{\tikz{\node[draw, line width=0.3mm, scale=0.9, star, star points=5, star point ratio=2, fill=Orange](){};}}}
\newcommand{\Rib}{\raisebox{0pt}{\tikz{\node[draw, line width=0.3mm, scale=0.9, star, star points=5, star point ratio=2, fill=Pink](){};}}}
%return

% Deoxynonulosonate
\newcommand{\Deoxynonulosonate}{\raisebox{0pt}{\tikz{\node[draw, line width=0.3mm, scale=1.1, diamond, fill=white](){};}}}
\newcommand{\Kdn}{\raisebox{0pt}{\tikz{\node[draw, line width=0.3mm, scale=1.1, diamond, fill=Green](){};}}}
\newcommand{\NeuAc}{\raisebox{0pt}{\tikz{\node[draw, line width=0.3mm, scale=1.1, diamond, fill=Purple](){};}}}
\newcommand{\NeuGc}{\raisebox{0pt}{\tikz{\node[draw, line width=0.3mm, scale=1.1, diamond, fill=LightBlue](){};}}}
\newcommand{\Neu}{\raisebox{0pt}{\tikz{\node[draw, line width=0.3mm, scale=1.1, diamond, fill=DarkBrown](){};}}}
\newcommand{\Sia}{\raisebox{0pt}{\tikz{\node[draw, line width=0.3mm, scale=1.1, diamond, fill=Red](){};}}}
%return

% Di-Deoxynonulosonate

% Unknown

% Assigned
\newcommand{\Assigned}{\raisebox{0.5pt}{\tikz{\node[draw, line width=0.3mm, scale=1.1, regular polygon, regular polygon sides=5,fill=white](){};}}}
\newcommand{\Api}{\raisebox{0.5pt}{\tikz{\node[draw, line width=0.3mm, scale=1.1, regular polygon, regular polygon sides=5,fill=Blue](){};}}}
\newcommand{\Fru}{\raisebox{0.5pt}{\tikz{\node[draw, line width=0.3mm, scale=1.1, regular polygon, regular polygon sides=5,fill=Green](){};}}}
\newcommand{\Tag}{\raisebox{0.5pt}{\tikz{\node[draw, line width=0.3mm, scale=1.1, regular polygon, regular polygon sides=5,fill=Yellow](){};}}}
\newcommand{\Sor}{\raisebox{0.5pt}{\tikz{\node[draw, line width=0.3mm, scale=1.1, regular polygon, regular polygon sides=5,fill=Orange](){};}}}
\newcommand{\Psic}{\raisebox{0.5pt}{\tikz{\node[draw, line width=0.3mm, scale=1.1, regular polygon, regular polygon sides=5,fill=Pink](){};}}}
%return


% You'll need to load the Tikz package in your preamble yourself if you do this.

% Author: Joe R. J. Healey
% Version: 1.0
% Email: J.R.J.Healey@warwick.ac.uk

% Changelog:
%  2018-05-16
%    > Initial creation and definition of some initial macros.
%     - Colours (RGB matches to NCBI documentation)
%     - Hexoses and N-acetyl hexoses

% To Do:
%  2018-05-16
%    > Replace hard coded values with variables and defaults.
%    > If the above is done, define a single macro for each glycan class (e.g. pentose)
%      and then make sub-macros of it for each individual sugar (e.g. \Ara{#1,#2,#3}

%%%%%%%%%%%%%%%%%%%%%%%%%%%%%%%%%%%%%%%%%%%%%%%%%%%%%%%%%%%%%%%%%%%%%%%%%%%%%%%%

\usepackage{tikz} 
\usetikzlibrary{angles,
                quotes,
                calc,
                positioning,
                shapes,
                arrows}


% Symbol Nomenclature for Glycans

% Colors (capitalised first letter to distinguish from inbuilt "green" etc.
\definecolor{Blue}{RGB}{0, 134, 248}
\definecolor{Green}{RGB}{0, 166, 91}
\definecolor{Yellow}{RGB}{255, 213, 65}
\definecolor{Orange}{RGB}{255, 118, 50}
\definecolor{Pink}{RGB}{255, 136, 192}
\definecolor{Purple}{RGB}{158, 58, 247}
\definecolor{LightBlue}{RGB}{129, 205, 230}
\definecolor{DarkBrown}{RGB}{148, 98, 41}
\definecolor{Red}{RGB}{255, 0, 24}

% Hexoses  (To do: replace some of the hard-coded values with vars and defaults.)
\newcommand{\Hexose}{\raisebox{0.5pt}{\tikz{\node[draw,line width=0.3mm, scale=1.1, circle, fill=white](){};}}}
\newcommand{\Glc}{\raisebox{0.5pt}{\tikz{\node[draw,line width=0.3mm, scale=1.1, circle, fill=Blue](){};}}}
\newcommand{\Man}{\raisebox{0.5pt}{\tikz{\node[draw,line width=0.3mm, scale=1.1, circle, fill=Green](){};}}}
\newcommand{\Gal}{\raisebox{0.5pt}{\tikz{\node[draw,line width=0.3mm, scale=1.1, circle, fill=Yellow](){};}}}
\newcommand{\Gul}{\raisebox{0.5pt}{\tikz{\node[draw,line width=0.3mm, scale=1.1, circle, fill=Orange](){};}}}
\newcommand{\Alt}{\raisebox{0.5pt}{\tikz{\node[draw,line width=0.3mm, scale=1.1, circle, fill=Pink](){};}}}
\newcommand{\All}{\raisebox{0.5pt}{\tikz{\node[draw,line width=0.3mm, scale=1.1, circle, fill=Purple](){};}}}
\newcommand{\Tal}{\raisebox{0.5pt}{\tikz{\node[draw,line width=0.3mm, scale=1.1, circle, fill=LightBlue](){};}}}
\newcommand{\Ido}{\raisebox{0.5pt}{\tikz{\node[draw,line width=0.3mm, scale=1.1, circle, fill=DarkBrown](){};}}}
%return

% N-Acetyl Hexoses
\newcommand{\HexNAc}{\raisebox{0.5pt}{\tikz{\node[draw,line width=0.3mm, scale=1.1, regular polygon, regular polygon sides=4, fill=white](){};}}}
\newcommand{\GlcNAc}{\raisebox{0.5pt}{\tikz{\node[draw,line width=0.3mm, scale=1.1, regular polygon, regular polygon sides=4, fill=Blue](){};}}}
\newcommand{\ManNAc}{\raisebox{0.5pt}{\tikz{\node[draw,line width=0.3mm, scale=1.1, regular polygon, regular polygon sides=4, fill=Green](){};}}}
\newcommand{\GalNAc}{\raisebox{0.5pt}{\tikz{\node[draw,line width=0.3mm, scale=1.1, regular polygon, regular polygon sides=4, fill=Yellow](){};}}}
\newcommand{\GulNAc}{\raisebox{0.5pt}{\tikz{\node[draw,line width=0.3mm, scale=1.1, regular polygon, regular polygon sides=4, fill=Orange](){};}}}
\newcommand{\AltNAc}{\raisebox{0.5pt}{\tikz{\node[draw,line width=0.3mm, scale=1.1, regular polygon, regular polygon sides=4, fill=Pink](){};}}}
\newcommand{\AllNAc}{\raisebox{0.5pt}{\tikz{\node[draw,line width=0.3mm, scale=1.1, regular polygon, regular polygon sides=4, fill=Purple](){};}}}
\newcommand{\TalNAc}{\raisebox{0.5pt}{\tikz{\node[draw,line width=0.3mm, scale=1.1, regular polygon, regular polygon sides=4, fill=LightBlue](){};}}}
\newcommand{\IdoNAc}{\raisebox{0.5pt}{\tikz{\node[draw,line width=0.3mm, scale=1.1, regular polygon, regular polygon sides=4, fill=DarkBrown](){};}}}
%return

% Hexosamine
% Half filled shapes are complicated, this will appear in a later update.

% Heuronate
% Half filled shapes are complicated, this will appear in a later update.

% Deoxyhexose (Note, numbers are not allowed in TeX macro command names, thus 6dTal becomes dTal)
\newcommand{\Deoxyhexose}{\raisebox{0.5pt}{\tikz{\node[draw, line width=0.3mm, scale=1.1, regular polygon, regular polygon sides=3,fill=white](){};}}}
\newcommand{\Qui}{\raisebox{0.5pt}{\tikz{\node[draw, line width=0.3mm, scale=1.1, regular polygon, regular polygon sides=3,fill=Blue](){};}}}
\newcommand{\Rha}{\raisebox{0.5pt}{\tikz{\node[draw, line width=0.3mm, scale=1.1, regular polygon, regular polygon sides=3,fill=Green](){};}}}
\newcommand{\dGul}{\raisebox{0.5pt}{\tikz{\node[draw, line width=0.3mm, scale=1.1, regular polygon, regular polygon sides=3,fill=Orange](){};}}}
\newcommand{\dAlt}{\raisebox{0.5pt}{\tikz{\node[draw, line width=0.3mm, scale=1.1, regular polygon, regular polygon sides=3,fill=Pink](){};}}}
\newcommand{\dTal}{\raisebox{0.5pt}{\tikz{\node[draw, line width=0.3mm, scale=1.1, regular polygon, regular polygon sides=3,fill=LightBlue](){};}}}
\newcommand{\Fuc}{\raisebox{0.5pt}{\tikz{\node[draw, line width=0.3mm, scale=1.1, regular polygon, regular polygon sides=3,fill=Red](){};}}}
%return

% DeoxyhexNAc

% Di-deoxyhexose

% Pentose
\newcommand{\Pentose}{\raisebox{0pt}{\tikz{\node[draw, line width=0.3mm, scale=0.9, star, star points=5, star point ratio=2, fill=white](){};}}}
\newcommand{\Ara}{\raisebox{0pt}{\tikz{\node[draw, line width=0.3mm, scale=0.9, star, star points=5, star point ratio=2, fill=Green](){};}}}
\newcommand{\Lyx}{\raisebox{0pt}{\tikz{\node[draw, line width=0.3mm, scale=0.9, star, star points=5, star point ratio=2, fill=Yellow](){};}}}
\newcommand{\Xyl}{\raisebox{0pt}{\tikz{\node[draw, line width=0.3mm, scale=0.9, star, star points=5, star point ratio=2, fill=Orange](){};}}}
\newcommand{\Rib}{\raisebox{0pt}{\tikz{\node[draw, line width=0.3mm, scale=0.9, star, star points=5, star point ratio=2, fill=Pink](){};}}}
%return

% Deoxynonulosonate
\newcommand{\Deoxynonulosonate}{\raisebox{0pt}{\tikz{\node[draw, line width=0.3mm, scale=1.1, diamond, fill=white](){};}}}
\newcommand{\Kdn}{\raisebox{0pt}{\tikz{\node[draw, line width=0.3mm, scale=1.1, diamond, fill=Green](){};}}}
\newcommand{\NeuAc}{\raisebox{0pt}{\tikz{\node[draw, line width=0.3mm, scale=1.1, diamond, fill=Purple](){};}}}
\newcommand{\NeuGc}{\raisebox{0pt}{\tikz{\node[draw, line width=0.3mm, scale=1.1, diamond, fill=LightBlue](){};}}}
\newcommand{\Neu}{\raisebox{0pt}{\tikz{\node[draw, line width=0.3mm, scale=1.1, diamond, fill=DarkBrown](){};}}}
\newcommand{\Sia}{\raisebox{0pt}{\tikz{\node[draw, line width=0.3mm, scale=1.1, diamond, fill=Red](){};}}}
%return

% Di-Deoxynonulosonate

% Unknown

% Assigned
\newcommand{\Assigned}{\raisebox{0.5pt}{\tikz{\node[draw, line width=0.3mm, scale=1.1, regular polygon, regular polygon sides=5,fill=white](){};}}}
\newcommand{\Api}{\raisebox{0.5pt}{\tikz{\node[draw, line width=0.3mm, scale=1.1, regular polygon, regular polygon sides=5,fill=Blue](){};}}}
\newcommand{\Fru}{\raisebox{0.5pt}{\tikz{\node[draw, line width=0.3mm, scale=1.1, regular polygon, regular polygon sides=5,fill=Green](){};}}}
\newcommand{\Tag}{\raisebox{0.5pt}{\tikz{\node[draw, line width=0.3mm, scale=1.1, regular polygon, regular polygon sides=5,fill=Yellow](){};}}}
\newcommand{\Sor}{\raisebox{0.5pt}{\tikz{\node[draw, line width=0.3mm, scale=1.1, regular polygon, regular polygon sides=5,fill=Orange](){};}}}
\newcommand{\Psic}{\raisebox{0.5pt}{\tikz{\node[draw, line width=0.3mm, scale=1.1, regular polygon, regular polygon sides=5,fill=Pink](){};}}}
%return


% You'll need to load the Tikz package in your preamble yourself if you do this.

% Author: Joe R. J. Healey
% Version: 1.0
% Email: J.R.J.Healey@warwick.ac.uk

% Changelog:
%  2018-05-16
%    > Initial creation and definition of some initial macros.
%     - Colours (RGB matches to NCBI documentation)
%     - Hexoses and N-acetyl hexoses

% To Do:
%  2018-05-16
%    > Replace hard coded values with variables and defaults.
%    > If the above is done, define a single macro for each glycan class (e.g. pentose)
%      and then make sub-macros of it for each individual sugar (e.g. \Ara{#1,#2,#3}

%%%%%%%%%%%%%%%%%%%%%%%%%%%%%%%%%%%%%%%%%%%%%%%%%%%%%%%%%%%%%%%%%%%%%%%%%%%%%%%%

\usepackage{tikz} 
\usetikzlibrary{angles,
                quotes,
                calc,
                positioning,
                shapes,
                arrows}


% Symbol Nomenclature for Glycans

% Colors (capitalised first letter to distinguish from inbuilt "green" etc.
\definecolor{Blue}{RGB}{0, 134, 248}
\definecolor{Green}{RGB}{0, 166, 91}
\definecolor{Yellow}{RGB}{255, 213, 65}
\definecolor{Orange}{RGB}{255, 118, 50}
\definecolor{Pink}{RGB}{255, 136, 192}
\definecolor{Purple}{RGB}{158, 58, 247}
\definecolor{LightBlue}{RGB}{129, 205, 230}
\definecolor{DarkBrown}{RGB}{148, 98, 41}
\definecolor{Red}{RGB}{255, 0, 24}

% Hexoses  (To do: replace some of the hard-coded values with vars and defaults.)
\newcommand{\Hexose}{\raisebox{0.5pt}{\tikz{\node[draw,line width=0.3mm, scale=1.1, circle, fill=white](){};}}}
\newcommand{\Glc}{\raisebox{0.5pt}{\tikz{\node[draw,line width=0.3mm, scale=1.1, circle, fill=Blue](){};}}}
\newcommand{\Man}{\raisebox{0.5pt}{\tikz{\node[draw,line width=0.3mm, scale=1.1, circle, fill=Green](){};}}}
\newcommand{\Gal}{\raisebox{0.5pt}{\tikz{\node[draw,line width=0.3mm, scale=1.1, circle, fill=Yellow](){};}}}
\newcommand{\Gul}{\raisebox{0.5pt}{\tikz{\node[draw,line width=0.3mm, scale=1.1, circle, fill=Orange](){};}}}
\newcommand{\Alt}{\raisebox{0.5pt}{\tikz{\node[draw,line width=0.3mm, scale=1.1, circle, fill=Pink](){};}}}
\newcommand{\All}{\raisebox{0.5pt}{\tikz{\node[draw,line width=0.3mm, scale=1.1, circle, fill=Purple](){};}}}
\newcommand{\Tal}{\raisebox{0.5pt}{\tikz{\node[draw,line width=0.3mm, scale=1.1, circle, fill=LightBlue](){};}}}
\newcommand{\Ido}{\raisebox{0.5pt}{\tikz{\node[draw,line width=0.3mm, scale=1.1, circle, fill=DarkBrown](){};}}}
%return

% N-Acetyl Hexoses
\newcommand{\HexNAc}{\raisebox{0.5pt}{\tikz{\node[draw,line width=0.3mm, scale=1.1, regular polygon, regular polygon sides=4, fill=white](){};}}}
\newcommand{\GlcNAc}{\raisebox{0.5pt}{\tikz{\node[draw,line width=0.3mm, scale=1.1, regular polygon, regular polygon sides=4, fill=Blue](){};}}}
\newcommand{\ManNAc}{\raisebox{0.5pt}{\tikz{\node[draw,line width=0.3mm, scale=1.1, regular polygon, regular polygon sides=4, fill=Green](){};}}}
\newcommand{\GalNAc}{\raisebox{0.5pt}{\tikz{\node[draw,line width=0.3mm, scale=1.1, regular polygon, regular polygon sides=4, fill=Yellow](){};}}}
\newcommand{\GulNAc}{\raisebox{0.5pt}{\tikz{\node[draw,line width=0.3mm, scale=1.1, regular polygon, regular polygon sides=4, fill=Orange](){};}}}
\newcommand{\AltNAc}{\raisebox{0.5pt}{\tikz{\node[draw,line width=0.3mm, scale=1.1, regular polygon, regular polygon sides=4, fill=Pink](){};}}}
\newcommand{\AllNAc}{\raisebox{0.5pt}{\tikz{\node[draw,line width=0.3mm, scale=1.1, regular polygon, regular polygon sides=4, fill=Purple](){};}}}
\newcommand{\TalNAc}{\raisebox{0.5pt}{\tikz{\node[draw,line width=0.3mm, scale=1.1, regular polygon, regular polygon sides=4, fill=LightBlue](){};}}}
\newcommand{\IdoNAc}{\raisebox{0.5pt}{\tikz{\node[draw,line width=0.3mm, scale=1.1, regular polygon, regular polygon sides=4, fill=DarkBrown](){};}}}
%return

% Hexosamine
% Half filled shapes are complicated, this will appear in a later update.

% Heuronate
% Half filled shapes are complicated, this will appear in a later update.

% Deoxyhexose (Note, numbers are not allowed in TeX macro command names, thus 6dTal becomes dTal)
\newcommand{\Deoxyhexose}{\raisebox{0.5pt}{\tikz{\node[draw, line width=0.3mm, scale=1.1, regular polygon, regular polygon sides=3,fill=white](){};}}}
\newcommand{\Qui}{\raisebox{0.5pt}{\tikz{\node[draw, line width=0.3mm, scale=1.1, regular polygon, regular polygon sides=3,fill=Blue](){};}}}
\newcommand{\Rha}{\raisebox{0.5pt}{\tikz{\node[draw, line width=0.3mm, scale=1.1, regular polygon, regular polygon sides=3,fill=Green](){};}}}
\newcommand{\dGul}{\raisebox{0.5pt}{\tikz{\node[draw, line width=0.3mm, scale=1.1, regular polygon, regular polygon sides=3,fill=Orange](){};}}}
\newcommand{\dAlt}{\raisebox{0.5pt}{\tikz{\node[draw, line width=0.3mm, scale=1.1, regular polygon, regular polygon sides=3,fill=Pink](){};}}}
\newcommand{\dTal}{\raisebox{0.5pt}{\tikz{\node[draw, line width=0.3mm, scale=1.1, regular polygon, regular polygon sides=3,fill=LightBlue](){};}}}
\newcommand{\Fuc}{\raisebox{0.5pt}{\tikz{\node[draw, line width=0.3mm, scale=1.1, regular polygon, regular polygon sides=3,fill=Red](){};}}}
%return

% DeoxyhexNAc

% Di-deoxyhexose

% Pentose
\newcommand{\Pentose}{\raisebox{0pt}{\tikz{\node[draw, line width=0.3mm, scale=0.9, star, star points=5, star point ratio=2, fill=white](){};}}}
\newcommand{\Ara}{\raisebox{0pt}{\tikz{\node[draw, line width=0.3mm, scale=0.9, star, star points=5, star point ratio=2, fill=Green](){};}}}
\newcommand{\Lyx}{\raisebox{0pt}{\tikz{\node[draw, line width=0.3mm, scale=0.9, star, star points=5, star point ratio=2, fill=Yellow](){};}}}
\newcommand{\Xyl}{\raisebox{0pt}{\tikz{\node[draw, line width=0.3mm, scale=0.9, star, star points=5, star point ratio=2, fill=Orange](){};}}}
\newcommand{\Rib}{\raisebox{0pt}{\tikz{\node[draw, line width=0.3mm, scale=0.9, star, star points=5, star point ratio=2, fill=Pink](){};}}}
%return

% Deoxynonulosonate
\newcommand{\Deoxynonulosonate}{\raisebox{0pt}{\tikz{\node[draw, line width=0.3mm, scale=1.1, diamond, fill=white](){};}}}
\newcommand{\Kdn}{\raisebox{0pt}{\tikz{\node[draw, line width=0.3mm, scale=1.1, diamond, fill=Green](){};}}}
\newcommand{\NeuAc}{\raisebox{0pt}{\tikz{\node[draw, line width=0.3mm, scale=1.1, diamond, fill=Purple](){};}}}
\newcommand{\NeuGc}{\raisebox{0pt}{\tikz{\node[draw, line width=0.3mm, scale=1.1, diamond, fill=LightBlue](){};}}}
\newcommand{\Neu}{\raisebox{0pt}{\tikz{\node[draw, line width=0.3mm, scale=1.1, diamond, fill=DarkBrown](){};}}}
\newcommand{\Sia}{\raisebox{0pt}{\tikz{\node[draw, line width=0.3mm, scale=1.1, diamond, fill=Red](){};}}}
%return

% Di-Deoxynonulosonate

% Unknown

% Assigned
\newcommand{\Assigned}{\raisebox{0.5pt}{\tikz{\node[draw, line width=0.3mm, scale=1.1, regular polygon, regular polygon sides=5,fill=white](){};}}}
\newcommand{\Api}{\raisebox{0.5pt}{\tikz{\node[draw, line width=0.3mm, scale=1.1, regular polygon, regular polygon sides=5,fill=Blue](){};}}}
\newcommand{\Fru}{\raisebox{0.5pt}{\tikz{\node[draw, line width=0.3mm, scale=1.1, regular polygon, regular polygon sides=5,fill=Green](){};}}}
\newcommand{\Tag}{\raisebox{0.5pt}{\tikz{\node[draw, line width=0.3mm, scale=1.1, regular polygon, regular polygon sides=5,fill=Yellow](){};}}}
\newcommand{\Sor}{\raisebox{0.5pt}{\tikz{\node[draw, line width=0.3mm, scale=1.1, regular polygon, regular polygon sides=5,fill=Orange](){};}}}
\newcommand{\Psic}{\raisebox{0.5pt}{\tikz{\node[draw, line width=0.3mm, scale=1.1, regular polygon, regular polygon sides=5,fill=Pink](){};}}}
%return

